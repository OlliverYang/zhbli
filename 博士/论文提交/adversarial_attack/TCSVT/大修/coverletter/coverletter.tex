\documentstyle{letter}
\addtolength{\voffset}{-0.5in}
\addtolength{\hoffset}{-0.3in}
\addtolength{\textheight}{2cm}
\begin{document}
\signature{Zhenbang Li, Yaya Shi, Jin Gao$^*$, Shaoru Wang, Bing Li, Pengpeng Liang, Weiming Hu}
            
\begin{letter}{}
\opening{\textbf{Dear Editors,}}

We would like to express our heartfelt gratitude to you and the reviewers for the insightful and helpful comments. When we revised the paper, we carefully considered and followed all the comments and suggestions provided by you and the reviewers. To summarize, we have made the following revisions:

(1) We have carefully considered and followed all the comments and suggestions related to the clarity of the writing, and made the new material a thoroughly revised manuscript. Specifically, several state-of-the-art anchor-free trackers have been discussed in Section II. The algorithm for the attack process has been added in Section III.C. We have explained in detail the perturbation update process in offline training. We have also added the advantages and limitations of the proposed method after experimental analysis and the future work to the conclusion. The descriptions of figures and tables have been rewritten, and the format of all the tables has been unified.

(2) The experiments have also been thoroughly re-evaluated in Section IV. Specifically, we have added more experiments to demonstrate the practicability of the attack method when the ground truth box information is missing in the training data. Besides the strategy to generate the fake trajectory following the real trajectory in Sec. III, we have also considered an alternative way to generate the fake trajectory. To analyze the impact of our perturbations, we have evaluated the attack performance when only adding perturbations on the template images or the search regions on GOT-Val. To further illustrate the effectiveness of our proposed method, we manually generate similar random patterns and add them on the template and search regions to show their attack performance. To reduce the perceptibility of the perturbations, we have examined a new strategy to perturb the search image.

(3) We have added 3 papers published in the IEEE Transactions on Circuits and Systems for Video Technology, which are most closely related to our manuscript, and analysed what is distinctive/new about our current manuscript related to these previously published papers.

We hope that our revised manuscript is now appropriate for publication in IEEE Transactions on Circuits and Systems for Video Technology. Specific responses to all the comments of each reviewer are included in response.pdf and highlighted using bold font after the comments of each reviewer for the convenience of cross-reference. To make the changes easier to identify where necessary, we also have underlined most of the revised parts in the manuscript and provide an underlined version for the convenience of second review.

\vspace{2\parskip}
\closing{Yours Sincerely,}
\vspace{5mm}
$^*$ Corresponding author:
\vspace{-2mm}

Name: Jin Gao   \quad\quad  E-mail: jin.gao@nlpr.ia.ac.cn
\end{letter}

\end{document}

