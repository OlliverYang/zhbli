%---------------------------------------------------------------------------%
%->> Backmatter
%---------------------------------------------------------------------------%
\chapter{作者简历及攻读学位期间发表的学术论文与研究成果}

\section*{作者简历}

\textbf{李振邦},男,汉族,1993 年 5 月生,山东省新泰市人

\textbf{联系方式}:zhenbang.li@nlpr.ia.ac.cn

\textbf{2016 年 9 月 至今} 硕博连读在读(免试):

	中国科学院自动化研究所~模式识别与智能系统专业

\textbf{2012年 9 月 至 2016 年 7 月} 工学学士:

	北京理工大学~计算机科学与技术专业~本科


\section*{已发表(或正式接受)的学术论文:}

{
\setlist[enumerate]{}% restore default behavior
\begin{enumerate}[nosep]
    \item \textbf{Zhenbang Li}, Qiang Wang, Jin Gao,  Bing Li, Weiming Hu, "Globally Spatial-Temporal Perception: a Long-Term Tracking System", \textit{International Conference on Image Processing} (\textbf{ICIP}), 2020.
    \item \textbf{Zhenbang Li}, Qiang Wang, Jin Gao,  Bing Li, Weiming Hu, "End-to-End Temporal Feature Aggregation for Siamese Trackers", \textit{International Conference on Image Processing} (\textbf{ICIP}), 2020.
    \item \textbf{Zhenbang Li}, Bing Li, Jin Gao, Liang Li, Weiming Hu, "Manipulating Template Pixels for Model Adaptation of Siamese Visual Tracking", \textit{IEEE Signal Processing Letters}, 2020.
\end{enumerate}
}

\section*{已投稿的学术论文:}

{
\setlist[enumerate]{}% restore default behavior
\begin{enumerate}[nosep]
    \item \textbf{Zhenbang Li}, Jin Gao, Yaya Shi, Bing Li, Pengpeng Liang, Weiming Hu, "Video-Agnostic Perturbations: Efficient Targeted Attacks for Siamese Visual Tracking", \textit{International Joint Conference on Artificial Intelligence} (\textbf{IJCAI}), 2021.
    \item Yaya Shi, Chunfeng Yuan, \textbf{Zhenbang Li}, Ziqi Zhang, Bing Li, Zheng-Jun Zha, Weiming Hu, "VTMScore: Evaluating Video Captioning via Video-Text Matching", \textit{The 50th Annual Meeting of the Association for Computational Linguistics} (\textbf{ACL}), 2021.
\end{enumerate}
}
\iffalse
\section*{申请或已获得的专利:}

(无专利时此项不必列出)

\section*{参加的研究项目及获奖情况:}

可以随意添加新的条目或是结构。
\fi
\chapter[致谢]{致\quad 谢}\chaptermark{致\quad 谢}% syntax: \chapter[目录]{标题}\chaptermark{页眉}
\thispagestyle{noheaderstyle}% 如果需要移除当前页的页眉
%\pagestyle{noheaderstyle}% 如果需要移除整章的页眉

时光荏苒,日月如梭,转眼间在自动化研究所度过了充满理想与奋斗的博士生涯。有太多让我难以忘却的点点滴滴,在此我也由衷地感谢那些曾经引导我、帮助我、激励我和支持我的老师、同学、朋友和家人!

首先,我要特别需要感谢的是我的导师胡卫明研究员。感谢胡老师给了我深造的机会,并为我创造了宽松、自由的科研环境,以及完善的基础设施,让我能够在温馨舒适的氛围中安心科研。另外,胡老师兢兢业业的治学态度,刻苦认真的工作状态,孜孜不倦的进取精神深深鼓舞着我,能够遇到这样的良师是我人生莫大的荣幸!

衷心感谢李兵老师。李老师在我的科研和生活上都给予了很大的关怀,为我提供了良好的科研资源。每当我在迷茫的时候,李老师总是在我的身旁无条件的支持我、鼓励我,给了我克服困难的勇气,让我重拾信心。

感谢课题组内的高晋老师、原春锋老师和罗冠老师对我的指导和帮助,同他们的讨论开阔了我的视野,提高了我的科研能力。特别是高晋老师,他对我的学术论文进行了细致的修正和指导,使我逐渐提升了对学术论文撰写的理解和认识,让我收益匪浅。



%感谢王强博士对我的帮助。在我博士课题没有思路的时候,或者不确定是否可行的时候,是他用自己独到的见解和对目标跟踪领域的独到的认识,告诉我这个idea可行或者不可行。总之在想法的初创阶段给了我重要的指导。他的见解是我在遇到问题时少走了许多弯路。在我遇到疑惑的时候为我指点迷津,他在计算机视觉领域的独特学术见地以及优秀的工程实践能力,使我在每次与他的交流过程中都能弥补自己的不足。

感谢研究生部的鞠召艳、曹娟、郭静、张志琳等老师对我的关心与帮助。感谢模式识别国家重点实验室综合办公室的连国臻、赵薇等老师,以及课题组内的徐秋艳秘书,正是有了你们的辛勤工作才有了我们良好的科研和学习环境。

感谢课题组里一起学习和工作过的李凯师兄、张梦丹师姐、杜杨师兄、杨浩师兄、王强师兄、周宗伟师兄,还有张子琦、张志鹏、王绍儒、姜君等师弟师妹们,能在这个团结友爱的大家庭和大家一起学习工作、朝夕相处,是我一生的宝贵财富。

感谢我的父母和姐姐。是他们的关怀和鼓励让我能够安心求学,顺利完成学业。他们的默默支持是我不断前进的动力。

最后,感谢评审论文的各位老师在百忙之中提出宝贵意见!

\vspace{8mm}
\rightline{2021 年春于北京}

\cleardoublepage[plain]% 让文档总是结束于偶数页,可根据需要设定页眉页脚样式,如 [noheaderstyle]
%---------------------------------------------------------------------------%
