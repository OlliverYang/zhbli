%---------------------------------------------------------------------------%
%->> Backmatter
%---------------------------------------------------------------------------%
\chapter{作者简历及攻读学位期间发表的学术论文与研究成果}

\textbf{本科生无需此部分}。

\section*{作者简历}

\textbf{李振邦},男,汉族,1993 年 5 月生,山东省新泰市人

\textbf{联系方式}:zhenbang.li@nlpr@ia.ac.cn

\textbf{2016 年 9 月 至今} 硕博连读在读(免试):

	中国科学院自动化研究所 模式识别与智能系统专业

\textbf{2012年 9 月 至 2016 年 7 月} 工学学士:

	北京理工大学 计算机科学与技术专业 本科


\section*{已发表(或正式接受)的学术论文:}

{
\setlist[enumerate]{}% restore default behavior
\begin{enumerate}[nosep]
    \item \textbf{Zhenbang Li}, Qiang Wang, Jin Gao,  Bing Li, Weiming Hu, "Globally Spatial-Temporal Perception: A Long-Term Tracking System", \textit{International Conference on Image Processing} (\textbf{ICIP}), 2020. (CCF C类)
    \item \textbf{Zhenbang Li}, Qiang Wang, Jin Gao,  Bing Li, Weiming Hu, "End-to-End Temporal Feature Aggregation For Siamese Trackers", \textit{International Conference on Image Processing} (\textbf{ICIP}), 2020. (CCF C类)
    \item \textbf{Zhenbang Li}, Bing Li, Jin Gao, Liang Li, Weiming Hu, "Manipulating Template Pixels for Model Adaptation of Siamese Visual Tracking", \textit{IEEE Signal Processing Letters}, 2020.
\end{enumerate}
}

\section*{申请或已获得的专利:}

(无专利时此项不必列出)

\section*{参加的研究项目及获奖情况:}

可以随意添加新的条目或是结构。

\chapter[致谢]{致\quad 谢}\chaptermark{致\quad 谢}% syntax: \chapter[目录]{标题}\chaptermark{页眉}
\thispagestyle{noheaderstyle}% 如果需要移除当前页的页眉
%\pagestyle{noheaderstyle}% 如果需要移除整章的页眉

时光荏苒,日月如梭,转眼间在自动化研究所度过了充满理想与奋斗的博士生涯。有太多让我难以忘却的点点滴滴,在此我也由衷地感谢那些曾经引导我、帮助我、激励我和支持我的老师、同学、朋友和家人!

首先,我要特别需要感谢的是我的导师胡卫明研究员。感谢胡老师给了我深造的机会,并为我创造了宽松、自由的科研环境,以及完善的基础设施,可以让我能够在温馨舒适的氛围中安心科研。另外,胡老师兢兢业业的治学态度,刻苦认真的工作状态,对科研事业的执着追求,生活中又和蔼可亲,他那孜孜不倦的进取精神深深鼓舞着我,值得我辈学习。

衷心感谢李兵老师。他始终十分关心我,在我的科研和生活上都给予了很大的关怀。曾经多次为我提供良好的科研资源,并尽自己的所能在科研上帮助我。每当我在迷茫的时候,李兵老师总是在我的身旁无条件的支持我,鼓励我,为我能够坚持下来带来了强大的动力。在我因科研受挫而迷茫彷徨时,胡老师能包容我的不足并给予悉心的帮助。李兵老师的关怀给了我克服苦难的勇气,是李兵老师的鼓励让我重拾信心。

衷心感谢高晋老师。高晋老师在目标跟踪领域的建树颇高。在论文写作方面,高老师帮我对所写的论文进行了认真细致的修改,促使我在论文写作方面有了较大的提高,并让我收益匪浅。严谨的治学作风值得我不断学习。

还要感谢课题组内的原春锋老师、罗冠老师对我的指导和帮助。

衷心感谢王强师兄。在我博士课题没有思路的时候,或者不确定是否可行的时候,是他用自己独到的见解和对目标跟踪领域的独到的认识,告诉我这个idea可行或者不可行。总之在想法的初创阶段给了我重要的指导。他的见解是我在遇到问题时少走了许多弯路。在我遇到疑惑的时候为我指点迷津,他在计算机视觉领域的独特学术见地以及强大的工程实践能力,使我在每次与他的交流过程中都能弥补自己的不足。

感谢研究生部的李磊、邸凌、鞠召艳、胡蓉、曹娟、郭静、张志琳等老师
对我的关心与帮助与关心。感谢模式识别国家重点实验室综合办公室的连国臻、赵薇等老师,以及课题组内的徐秋艳秘书,正是有了你们的辛勤工作才有了我们良好的科研和学习。

感谢课题组里一起学习和工作过的李凯师兄、张梦丹师姐、杜杨师兄、杨浩师兄、王强师兄、周宗伟师兄,还有张子琦、张志鹏、杨力、王绍儒、姜君等师弟师妹们,能在这个团结友爱的大家庭和大家一起学习工作、朝夕相处,是我一生的宝贵财富。

感谢我的父母和姐姐。是你们的支持和鼓励让我能够安心求学,顺利完成学业。他们的默默支持是我不断前进的动力。

最后,感谢评审论文的各位老师在百忙之中提出宝贵意见!
\cleardoublepage[plain]% 让文档总是结束于偶数页,可根据需要设定页眉页脚样式,如 [noheaderstyle]
%---------------------------------------------------------------------------%
