\chapter{引言}\label{chap:introduction}
视频目标跟踪是计算机视觉领域的一个重要研究课题,它以摄像机拍摄得到的序列图像为研究对象,其中包含与摄像机有相对运动关系的目标,以在视频的连续帧之间创建基于位置、速度、形状、纹理、色彩等有关特征的对应匹配为研究目的,建立目标在帧间的关联关系。视觉目标跟踪问题经过数十年的发展,逐步形成了以基础应用需求不同而衍生的多种特定类别目标跟踪研究。针对跟踪任意运动目标的通用跟踪问题而广泛开展的模型非固定式在线视觉跟踪研究(model-free tracking,MFT);针对监控场景下广泛存在的特定运动目标(如行人、车辆等),而开展的基于特定模型检测的离线/在线多目标跟踪研究(model-based tracking,MBT);针对室内场景下类似于Microsoft Kinect这类RGBD摄像机的应用场景开展的基于深度(depth)信息的RGBD跟踪研究;以及针对机器人自主导航应用中的视觉里程计(visual odometry,VO)开展的基于关键特征点匹 配的相关跟踪算法研究等等。本文主要针对MFT展开研究。
\section{研究背景与意义}
\begin{itemize}
\item 智能监控。智能监控的目的是由计算机智能地分析摄像头所获取
的图像序列,对场景内容进行理解,实现对异常行为的自动报警和预警。
智能视觉监控技术大概可以分为底层视觉和高层视觉两个部分。底层视觉主要是对场景中感兴趣目标进行检测、跟踪和识别,而高层视觉则是在底层视觉的基础上对感兴趣目标进行行为分析和理解。智能视觉监控技术可以广泛应用于公共安全监控、工业现场监控、居民小区监控、交通状态监控等各种监控场景中,能够显著提高监控效率,降低监控成本,具有广泛的研究意义和应用前景。视觉跟踪的任务就是在连续的图像序列中对运动目标进行定位,为后续行为分析提供目标轨迹和运动参数等信息。它在智能视觉监控技术中起着至关重要的作用。跟踪所提供的丰富的时间和空间信息,可以使目标检测和识别更加鲁棒;同时跟踪所提供的轨迹信息和运动参数是目标行为分析和理解的基本前提。

% http://www.ryxxff.com/67630.html https://arxiv.org/pdf/1704.05519.pdf
\item 自动驾驶。自20世纪80年代中期以来,世界各地的许多大学、研究中心、汽车公司和其他行业的公司都在研究和开发自动驾驶汽车(又称无人驾驶汽车和自动驾驶汽车)。自动驾驶汽车自主系统的体系结构一般分为感知系统和决策系统。感知系统一般分为多个子系统,负责自驾汽车定位、静态障碍物绘制、移动障碍物检测与跟踪、道路绘制、交通信号检测与识别等任务。其中,利用车载传感器采集的数据(如光探测和测距(LiDAR)、无线电探测和测距(雷达)、摄像头、全球定位系统(GPS)、惯性测量(IMU)、里程计等多模态信息,对目标进行识别和跟踪,便是重要一环。跟踪其他交通参与者是自动驾驶的一项非常重要的任务。例如,考虑车辆的制动距离,该制动距离随其速度成倍增加。由于制动距离,有必要及早发现与其他交通参与者的碰撞。只有对未来的轨迹做出良好的预测,这才有可能。对于行人和骑自行车的人来说,预测未来的行为尤其困难,因为他们会突然改变他们的运动方向。因此,人们倾向于在行人和骑自行车的人周围更加谨慎地驾驶。类似地,结合交通参与者的分类进行跟踪允许相应地调整车辆的速度。此外,对其他车辆的跟踪可用于自动距离控制,并提前预测其他交通参与者的可能驾驶行为(例如接管)。跟踪系统必须应对各种挑战,例如背景杂乱,运动的多样性和复杂性以及遮挡。由于不同对象(尤其是同一类)的相似性,随着时间的推移将同一对象的实例关联在一起的问题变得特别具有挑战性。除了由于与其他对象的相似性而缺乏区分性信息之外,同一对象的实例可能看起来不够相似,无法在不同的时间步长进行关联。通常,对象被其他对象或其自身部分或完全遮挡。对象之间的相互作用,特别是在行人的情况下,进一步增加了遮挡的数量,并使得难以跟踪每个单独的对象。困难的照明条件和镜子或窗户的反射带来了其他挑战。因此,设计优秀的跟踪器对于自动驾驶而言是十分重要的任务。
\end{itemize}

\subsection{计算机视觉}
计算机视觉是计算机科学与模式识别领域的一项重要的科学研究任务。该任务的目的是使用计算机对数字图像、视频进行智能地自动化分析,实现对数字图像/视频内容的智能感知和理解。如对对图像/视频中目标的识别、检测、跟踪、分割等。视频目标跟踪便是计算机视觉中的一项重要研究课题。视频目标跟踪指对视频中出现的任一物体进行跟踪:首先用一个矩形框在视频的第一帧中框住感兴趣的物体,视频目标跟踪算法需要准确高效地在后续每一帧中都预测一个矩形框,用于表示感兴趣物体的位置、尺寸和长宽比。
